\documentclass[11pt,timesnewroman,partone,draftproposal]{erc-stg}

% {{{ packages

% better environments
\usepackage[nameinlink, noabbrev]{cleveref}
\usepackage{booktabs}
\usepackage{caption}
\usepackage{subcaption}

% better typography
\usepackage{microtype}

% }}}

% {{{ title

% \ercauthor{John Doe}
% \ercauthorlastname{Doe}
% \ercinstitution{Massachusetts Institute of Technology (MIT)}
% \ercduration{60 months}
%
% \erctitle{Quantum Understanding and Simulated Analysis Research}
% \ercacronym{QUASAR}

% }}}

\begin{document}

% \ercsummary{}

\maketitle

\section{Extended Synopsis of the Scientific Proposal (max. 5 pages, references
do not count towards the page limit)}
\label{sc:synopsis}

\begin{erccomment}
[The Extended Synopsis should give a concise presentation of the scientific proposal,
with particular attention to the ground-breaking nature of the research project,
which will allow evaluation panels to assess, in Step 1 of the evaluation, the
feasibility of the outlined scientific approach. Describe the proposed work in
the context of the state of the art of the field. It is important that the extended
synopsis contains minimum information relevant to the evaluation criteria, since
the \textbf{Step 1 panel will have access only to part B1}.

References to literature should also be included. Please use a reference style
that is commonly used in your discipline such as American Chemical Society (ACS)
style, American Medical Association (AMA) style, Modern Language Association
(MLA) style, etc. and that allows the evaluators to easily retrieve each reference.]

\textbf{Please respect the following formatting constraints: Times New Roman,
Arial or similar, at least font size 11, margin sizes (2.0 cm side and 1.5 cm top
and bottom), single line spacing.}
\end{erccomment}

\newpage

\section{Curriculum Vitae and Track Record (max. 4 pages)}
\label{sc:sv}

\begin{erccomment}
[You may modify the below template if necessary.]
\end{erccomment}

\subsection*{PERSONAL DETAILS}
\label{ssc:cv:details}

\begin{erccomment}
[Provide your personal details, your education and key qualifications, current
position(s) and relevant previous positions you have held.] \\
\end{erccomment}

Family name, First name:

Researcher unique identifier(s) (such as ORCiD, Research ID, etc.):

URL for web site:

\subsubsection*{Education and key qualifications}
\label{ssc:cv:education}

\begin{erccvitem}
DD/MM/YYYY & PhD \\
& Name of Faculty / Department, Name of University / Institution, Country, \\
& \underline{Name of PhD advisor}.
\end{erccvitem}
\medskip

\begin{erccvitem}
YYYY & Master \\
& Name of Faculty / Department, Name of University / Institution, Country.
\end{erccvitem}

\subsubsection*{Current position(s)}

\begin{erccvitem}
YYYY -- YYYY & Current position \\
& Name of Faculty / Department, Name of University / Institution, Country.
\end{erccvitem}
\medskip

\begin{erccvitem}
YYYY -- YYYY & Current position \\
& Name of Faculty / Department, Name of University / Institution, Country.
\end{erccvitem}

\subsubsection*{Previous position(s)}

\begin{erccvitem}
YYYY -- YYYY & Position held \\
& Name of Faculty / Department, Name of University / Institution, Country.
\end{erccvitem}
\medskip

\begin{erccvitem}
YYYY -- YYYY & Position held \\
& Name of Faculty / Department, Name of University / Institution, Country.
\end{erccvitem}

\subsection*{RESEARCH ACHIEVEMENTS AND PEER RECOGNITION}
\label{ssc:cv:research}

\subsubsection*{Research Achievements}

\begin{erccomment}
[Provide a list of up to ten research outputs that demonstrate how you have advanced
knowledge in your field with an emphasis on more recent achievements, such as
publications, articles deposited in a publicly available preprint server, books,
book chapters, conference proceedings, data sets, software, patents, licenses,
standards, start-up businesses or any other research outputs you deem relevant
in relation to your research field and your project.

You may include a short, factual explanation of the significance of the selected
outputs, your role in producing each of them, and how they demonstrate your
capacity to successfully carry out your proposed project.]
\end{erccomment}

\subsubsection*{Peer Recognition}

\begin{erccomment}
[Provide a list of selected examples of significant recognition by your peers
if applicable, such as prizes, awards, fellowships, elected academy memberships,
invited presentations to major conferences or any other examples of significant
recognition you deem relevant in relation to your research field and project.

You may include a short explanation of the significance of the listed examples.]
\end{erccomment}

\subsection*{ADDITIONAL INFORMATION}
\label{ssc:cv:additional_information}

\begin{erccomment}
You may provide relevant additional information on your research career to
provide context to the evaluation panels when assessing your research achievements
and peer recognition as described above.
\end{erccomment}

\subsubsection*{Career breaks, diverse career paths, and major life events}

\begin{erccomment}
[You may include a short factual explanation of career breaks or diverse career
paths such as secondments, volunteering, part-time work, time spent in different
sectors or the effects of major life events such as long term illness as well as
the effects of pandemic restrictions on research productivity.]
\end{erccomment}

\subsubsection*{Other contributions to the research community}

\begin{erccomment}
[You may include a list of particularly noteworthy contributions to the research
community you have made other than research achievements and peer recognition and
a short explanation of these contributions. The purpose of this section is to
allow the panels to take a more rounded view of your career and achievements and
to ensure that any additional responsibilities, commitments and leadership roles
that you have taken on beyond your individual research activities are recognised
and taken into account.]

\textbf{
[(for more information see ``Information for Applicants to the Starting and
Consolidator Grant 2025 Calls'')]
}

\textbf{Do NOT split the sections and/or references in Part B1 and do NOT upload
them as separate documents. The peer reviewers will only receive one single
document for evaluation at Step 1. Hence, Part B1 should contain all elements as
explained in this template. If some parts of Part B1 are uploaded in the submission
system as separate attachments, the peer reviewers will not have access to them.}
\end{erccomment}

\end{document}

% kate: default-dictionary en_GB;
